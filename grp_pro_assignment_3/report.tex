\documentclass[a4paper,11pt]{article}

\usepackage[T1]{fontenc}
\usepackage[utf8]{inputenc}
\usepackage{graphicx}
\usepackage{xcolor}

\renewcommand\familydefault{\sfdefault}
\usepackage{tgheros}
\usepackage[defaultmono]{droidmono}

\usepackage{amsmath,amssymb,amsthm,textcomp}
\usepackage{enumerate}
\usepackage{multicol}
\usepackage{tikz}
\usetikzlibrary{arrows}

\usepackage{geometry}
\geometry{total={210mm,297mm},
left=25mm,right=25mm,%
bindingoffset=0mm, top=20mm,bottom=20mm}


\linespread{1.3}

\newcommand{\linia}{\rule{\linewidth}{0.5pt}}

% custom theorems if needed
\newtheoremstyle{mytheor}
    {1ex}{1ex}{\normalfont}{0pt}{\scshape}{.}{1ex}
    {{\thmname{#1 }}{\thmnumber{#2}}{\thmnote{ (#3)}}}

\theoremstyle{mytheor}
\newtheorem{defi}{Definition}

% my own titles
\makeatletter
\renewcommand{\maketitle}{
\begin{center}
\vspace{2ex}
{\huge \textsc{\@title}}
\vspace{1ex}
\\
\linia\\
\@author \hfill \@date
\vspace{4ex}
\end{center}
}
\makeatother
%%%

% custom footers and headers
\usepackage{fancyhdr}
\pagestyle{fancy}
\lhead{}
\chead{}
\rhead{}
\lfoot{Assignment \textnumero{} 3}
\cfoot{}
\rfoot{Page \thepage}
\renewcommand{\headrulewidth}{0pt}
\renewcommand{\footrulewidth}{0pt}
%

% code listing settings
\usepackage{listings}
\lstset{
    language=Python,
    basicstyle=\ttfamily\small,
    aboveskip={1.0\baselineskip},
    belowskip={1.0\baselineskip},
    columns=fixed,
    extendedchars=true,
    breaklines=true,
    tabsize=4,
    prebreak=\raisebox{0ex}[0ex][0ex]{\ensuremath{\hookleftarrow}},
    frame=lines,
    showtabs=false,
    showspaces=false,
    showstringspaces=false,
    keywordstyle=\color[rgb]{0.627,0.126,0.941},
    commentstyle=\color[rgb]{0.133,0.545,0.133},
    stringstyle=\color[rgb]{01,0,0},
    numbers=left,
    numberstyle=\small,
    stepnumber=1,
    numbersep=10pt,
    captionpos=t,
    escapeinside={\%*}{*)}
}

\tikzstyle{arn_n} = [circle, white, font=\bfseries, draw=black, fill=black, align=center, inner sep=0pt,text width=1.5em, text centered]% black node
\tikzstyle{arn_r} = [circle, red, draw=red, align=center, inner sep=0pt,text width=1.5em, text centered, very thick]% red node
\tikzstyle{arn_x} = [rectangle, draw=black, align=center, inner sep=0pt,minimum width=0.5em, minimum height=0.5em]% NIL 'node'

%%%----------%%%----------%%%----------%%%----------%%%

\begin{document}

\title{Programming Assignment \textnumero{} 3}

\author{Harsh Sinha(14265), Deepak Gangwar(14208)}

\date{01/04/2017}

\maketitle

\section*{Problem 1}
\subsection*{Part A}
\begin{center}
 \begin{tabular}{||c c c c c c||} 
 \hline
  & $n = 10^2$ & $n = 10^3$ & $n = 10^4$ & $n = 10^5$ & $n = 10^6$\\ [0.5ex] 
 \hline\hline
 Average running time(Sec.) \\of QuickSort & $1.43 * 10^-5$ & $1.23 * 10^-4$ & $1.27 * 10^-3$ & $1.53 * 10^-2$ & $0.185568$\\ 
 \hline
 Average running time(Sec.) \\of MergeSort & $1.97 * 10^-5$ & $1.62 * 10^-4$ & $1.66 * 10^-3$ & $1.95 * 10^-2$ & $0.235445$\\
 \hline
 Average number of \\comparisons in
QuickSort & 353 & 5753 & 79443 & 1027068 & 12490416\\
 \hline
 Average number of \\comparisons in
MergeSort & 267 & 4346 & 60191 & 768229 & 9337462\\
 \hline
 No. of times MergeSort \\had lesser no.
of comparisons \\than QuickSort & 938 & 999 & 1000 & 1000 & 1000\\ [1ex] 
 \hline
\end{tabular}
\end{center}

The above table is obtained by avaraging the execution time for 1000 loops, where the numbers in the array are pseudo-randomly generated.
The comparisions for merge sort are counted in the mergeAarray function while comparing the values from the temporary arrays and the merging them; This for quick sort is done when the numbers are compared inside the partition function.
The algorithm for merge sort takes lesser number of comparisions than that for quick sort for most of the values, except for the smaller ones.
Interestingly though merge sort takes higher amount of avarage time for execution on the same array for all of the array sizes.
Now this is due to the fact that 


\subsection*{Part B}
\begin{center}
 \begin{tabular}{||c c c c c c||} 
 \hline
  & $n = 10^2$ & $n = 10^3$ & $n = 10^4$ & $n = 10^5$ & $n = 10^6$\\ [0.5ex] 
 \hline\hline
 Average running time(Sec.) \\of QuickSort & $1.43 * 10^-5$ & $1.23 * 10^-4$ & $1.27 * 10^-3$ & $1.53 * 10^-2$ & $0.185568$\\ 
 \hline
 Average number of \\comparisons in
QuickSort & 353 & 5797 & 80443 & 1034672 & 12490416\\
 \hline
Percentage of cases \\when running time
of QuickSort \\exceeds average by 5\% & 651 & 699 & 928 & 976 & 956\\
 \hline
 Percentage of cases \\when running time
of QuickSort \\exceeds average by 10\% & 68 & 34 & 15 & 20 & 32\\
 \hline
 Percentage of cases \\when running time
of QuickSort \\exceeds average by 20\% & 193 & 63 & 10 & 3 & 7\\
 \hline
 Percentage of cases \\when running time
of QuickSort \\exceeds average by 30\% & 57 & 45 & 14 & 0 & 3\\
 \hline
 Percentage of cases \\when running time
of QuickSort \\exceeds average by 40\% & 14 & 38 & 13 & 0 & 1\\
 \hline
 Percentage of cases \\when running time
of QuickSort \\exceeds average by 50\% & 3 & 27 & 3 & 0 & 1\\
 \hline
 Percentage of cases \\when running time
of QuickSort \\exceeds average by 100\% & 11 & 79 & 11 & 0 & 0\\
 \hline
\end{tabular}
\end{center}

\p{For each value of n, average values and exceed percentages have been determined for 1000 iterations. It can be seen from average run time and average comparison values of Quick sort that it is of the order $O(nlogn)$.\\ As array size increases, run time values for each different array gets closer to average for that array size. Also, for same array size, run time values exceeding a certain percentage decreases asymptotically.}

\section*{Problem 2}
\subsection*{Part A}
\textbf{Pseudo Code: }
\begin{lstlisting}[label={list:first},caption=Op, mathescape = true]
void Op (int c[],int low, int high)
{
	n $\leftarrow$ high - low + 1; //number of elements between high and low index
	if(n <= 1) //base case
	    return;
    two[],one[],zero[];
    for (i = low,i < high,i++){
        if(i-low < size(two)) c[i] = two[i - low];
        if(i-low < size(one)) c[i] = one[i - low];
        if(i-low < size(zero)) c[i] = zero[i - low];
    }
    c[] = two[] + one[] + zero[]; //in O(n)
    free(two);free(one);free(zero);
    Op(c, low , low+size(two)-1);
    Op(c, low+size(two) , low+size(two)+size(one)-1);
    Op(c, low+size(two)+size(one) , low+size(two)+size(one)+size(zero)-1);
}
\end{lstlisting}
\begin{lstlisting}[label={list:first},caption=Query, mathescape = true]
c = Op(c,0,n-1); //c is the input array of Size n
b[n+1];
b[0] $\leftarrow$ 0;
	for(int64_t i = 0; i < n;i++){
		b[i+1] = b[i] + c[i];
	}
void Query(i,j,b){
    return b[j+1] - b[i];
}
\end{lstlisting}

\textbf{Time Complexity: }\\
Let the size of input array is \textbf{n} and number of queries is \textbf{q}.\\
\[T_{Op}(n) = cn + 3 * T_{Op}(n/3) \textit{\ and\ } T_{Op}(1) = 1 \]
\[\implies T_{Op}(n) = O(nlogn)\]\\
Preprocessing for Query is O(n).\\
\[T_{Query} = q * O(1)\\\]
\[\implies T_{Query} = O(q)\]\\

\textbf{Space Complexity: }\\
in Op three arrays are made with total size of n but they are being freed before the recursive call so extra space needed is of order n.\\
\textbf{Space Complexity of Op = O(n)}\\
similarly for Query a new array is maintained.\\
So, \textbf{Space Complexity of Query = O(n)}\\
\end{document}
